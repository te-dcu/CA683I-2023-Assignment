The World Health Organisation released a sobering report at the end of 2018, and it highlighted that road traffic deaths has reached 1.35 million per year \cite{who2018}.
Road traffic related deaths have become the main cause of death of people aged between 5 and 29 years and the eight leading cause for all age groups, surpassing HIV/AIDS \cite{who2018}.
The UK alone reported an estimated 1,560 road deaths for 2021 \cite{GBRTE2021}.
With people travelling more miles in Great Britain increasing year-on-year, accidents and their prevention become a big concern for policymakers \cite{GBRTE2021}\footnote{Due to the coronavirus (COVID-19) pandemic, the report's long term trends for 2021 can be misleading.}.

This study seeks to identify factors that contribute to serious road accidents using the Road Traffic Accident Data \cite{RTAD2023} provided by the UK Department for Transport for the period of 2005--2017 containing 2,058,408 incidents.
A serious accident is defined as incidents requiring hospitalisation or resulting in fatalities.

\begin{itemize}
    \item \textbf{The Null hypothesis ($H0$)}: There is no significant relationship between the factors under consideration and the severity of accidents.
    \item \textbf{The Alternative hypothesis ($H1$)}: There exists a significant relationship between at least one of the factors under consideration and the severity of accidents.
\end{itemize}

In order to assess the significance of the factors, we will use $\alpha=0.05$ for the null hypothesis ($H0$).
This $\alpha$ value represents a 95\% level of confidence that we wish to achieve before a factor rejects the null hypothesis.

By exploring the relationships between the factors and accident severity, the study aims to contribute to existing knowledge and provide insights that can help enhance road safety measures and potentially save lives.

Section \ref{sec:related-work} examines existing literature on different road accident risk factors and the methods used for analysing them.
Section \ref{sec:methodology} outlines the methodology employed by this paper.
Section \ref{sec:evaluation/results} presents the findings, while section \ref{sec:conclusions-and-future-work} offers concluding remarks and future work.
