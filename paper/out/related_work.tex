A review of existing literature was conducted to observe different methods and techniques that were applied to study factors that increase the risk of serious road crashes.

\citetitle{VALENT200271} \cite{VALENT200271} utilised data from regional authorities and employed statistical analysis techniques to identify significant factors that contributed to fatal road accidents.
Lack of wearing a seatbelt, speeding, age cohort, poor visibility were found as contributing factors.
The study used logistic regression to evaluate a connection between driver attributes and the circumstances of severe road traffic accidents.
However, the paper acknowledged that there were limitations in the data that was collected, in particular information about seat belt usage and alcohol and drug usage.

The next paper reviewed used a similar dataset that this paper used.
\citetitle{10.1007/978-3-030-39431-8_50} \cite{10.1007/978-3-030-39431-8_50} applied the Apriori algorithm used in Association Rule Mining approach along with network visualisation to analyse UK road accident data from 2005 to 2017.
Potential further studies regarding the method used by the paper can look closer to see if the rules change significantly between cities and also how it would apply to crime.


\citetitle{MICHALAKI201589} \cite{MICHALAKI201589} utilised UK road traffic accident data between 2005 and 2011, but with a focus on accidents that took place on the hard-shoulder of the motorways.
The study used Generalised Ordered Logistic Regression (logit) to analyse how certain factors influenced the likelihood of three ranked accidents (Fatal, Severe, Slight) occurring.
Factors found to increase the severity of the accident involved the hour of the day when there is high levels of traffic, visibility, and fatigue in Heavy Goods Vehicle (HGV) drivers.



\citetitle{kumar2015datamining} \cite{kumar2015datamining} examined road accidents in a small Indian region, proposing a two-phase process using K-Modes clustering and association rule mining for each cluster and the entire dataset.
They identified six distinct clusters and their specific characteristics.

\citetitle{COMI2022798} \cite{COMI2022798} analysed Rome Municipality road accident data using K-Means clustering, Kohonen networks, decision trees, association rule mining, and artificial neural networks, concluding that vehicle type significantly influences accident severity.


\citetitle{Nour2020} \cite{Nour2020} looks at data from a similar source as we do and for a similar period.
It combines three elements - Accident information, Vehicle information and Casualty information.
Interestingly, many categorical features with multiple possible values were reduced to binary variables.
The paper examines the use of XGBoost the most accurate at 74.4\% to determine what are the key factors contributing to severity.
They concluded that Casualty Type has the greatest bearing on accident severity.



 \citetitle{ijerph17114111} \cite{ijerph17114111} examines the problem of fatal road collisions.
Using machine learning methods, the study tries to uncover the variables related with deadly road collisions.
The study looked at the relationship between road features, driver-related variables, vehicle types, and meteorological conditions and the occurrence of fatal road collisions.
The outcomes of the research demonstrated that various variables, including driver-related factors such as speeding, distracted driving, and driving while impaired by drugs or alcohol, were substantially correlated with fatal road collisions.


\citetitle{rjrsmsfa} \cite{rjrsmsfa} employed a mixed-methods approach to study road accidents, revealing key contributing factors despite the subjective nature of survey and interview data.
The study underscores the importance of effective road safety measures and implementing appropriate data mining methods like association rule mining.


\citetitle{computers11050080} \cite{computers11050080} compares the effectiveness of several statistical and machine learning models for predicting the severity of traffic accidents.
The study emphasises the need for precise accident severity categorisation to help improve road safety and reduce deaths.
Machine learning approaches outperformed traditional statistical models in predicting accident severity.

\citetitle{7965753} \cite{7965753} utilised data processing techniques to analyse a large dataset concerning road accidents.
They examined the correlation between fatality rates and various factors, including collision types, weather conditions, road surface conditions, lighting conditions, and the presence of intoxicated drivers.
They employed the Apriori algorithm to identify association rules, designed a classification model using the Naive Bayes classifier, and created clusters using the simple K-means clustering method.


\citetitle{IRJET-V7I51291} \cite{IRJET-V7I51291} extracted frequent road accident patterns using association and classification rules.
They used the Apriori algorithm to analyse historical data and predict accident types on both existing and new roads.

\citetitle{Karthik2019IdentifyingER} \cite{Karthik2019IdentifyingER} studied road accidents with data mining techniques, employing algorithms like Naive Bayes, Random Forest, and J48.
Their analysis aimed to uncover incident reasons, pinpoint high-risk areas, and identify leading causes of severe accidents.