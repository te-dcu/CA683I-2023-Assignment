This study examined UK road traffic accidents between 2005 and 2017, focusing on determining the factors significantly related to accident severity.
Out of a total of 2,058,408 incidents, 14\% were classified as severe accidents.
The study employed a Chi-square test to analyze the relationship between 219 features and accident severity, discovering that 198 features rejected the null hypothesis, thus indicating a significant relationship with accident severity.

The most influential factors included vehicle type (motorcycles being particularly prominent), speed limits (specifically 60 miles per hour), the first point of impact, various vehicle manoeuvres (such as waiting to go or slowing down), light conditions (darkness with no lighting), vehicles leaving the carriageway (either nearside or offside), hitting objects off the carriageway (notably trees), and junction location (not at or within 20 meters of a junction).
It is important to note that these factors do not imply causation but rather a higher prevalence of severe accidents compared to less severe ones.

To optimise the prediction of accident severity using a machine learning model, Recursive Feature Elimination (RFE) was utilised.
This method aimed to identify the ideal balance between the number of features and model performance.
The RFE process pinpointed 43 features as the optimal balance, suggesting that incorporating these top factors in the model is likely to yield the most accurate predictions for accident severity.